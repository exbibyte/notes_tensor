\documentclass[8pt,letter]{article}
 
%% \usepackage[fleqn]{amsmath}
\usepackage[margin=1in]{geometry}
\usepackage{amsmath,amsfonts,amsthm,bm}
\usepackage{breqn}
\usepackage{amsmath}
\usepackage{mathtools}
\usepackage{amssymb}
\usepackage{tikz}
\usepackage[ruled,vlined,linesnumbered,lined,boxed,commentsnumbered]{algorithm2e}
\usepackage{siunitx}
\usepackage{graphicx}
\usepackage{subcaption}
%% \usepackage{datetime}
\usepackage{multirow}
\usepackage{multicol}
\usepackage{mathrsfs}
\usepackage{fancyhdr}
\usepackage{fancyvrb}
\usepackage{parskip} %turns off paragraph indent
\pagestyle{fancy}

\usepackage{xcolor}
\usepackage{mdframed}

\usepackage[small]{titlesec}

\usetikzlibrary{arrows}

\DeclareMathOperator*{\argmin}{argmin}
\newcommand*{\argminl}{\argmin\limits}

\newcommand{\mathleft}{\@fleqntrue\@mathmargin0pt}
\newcommand{\R}{\mathbb{R}}
\newcommand{\Z}{\mathbb{Z}} 
\newcommand{\N}{\mathbb{N}}
\newcommand{\ppartial}[2]{\frac{\partial #1}{\partial #2}}
\newcommand{\p}{\partial}
\newcommand{\te}[1]{\text{#1 }}
\newcommand{\norm}[1]{\|#1\|}

\setcounter{MaxMatrixCols}{20}

% remove excess vertical space for align equations
\setlength{\abovedisplayskip}{0pt}
\setlength{\belowdisplayskip}{0pt}
\setlength{\abovedisplayshortskip}{0pt}
\setlength{\belowdisplayshortskip}{0pt}

\newtheorem{theorem}{Theorem}[section]
\newtheorem{corollary}{Corollary}[theorem]
\newtheorem{lemma}[theorem]{Lemma}

% \newtheorem{mdtheorem}{Theorem}
% \newenvironment{theorem}
% {\begin{mdframed}[
%   backgroundcolor=green!10,
%   topline=false,
%   rightline=false,
%   bottomline=false,
%   leftline=false
%   ]\begin{mdtheorem}}
%     {\end{mdtheorem}\end{mdframed}}

\begin {document}

\lhead{Notes - Tensor Calculus, Bill (Yuan) Liu}

% \begin{align}
%   \nabla f_{k+1}^T p_k & \geq c_2 \nabla f_k^T p_k\\
%   \frac{\partial \phi(\alpha_k)}{\partial \alpha_k} & \geq \frac{\partial \phi(0)}{\partial \alpha_k}\\
%   &c_1,\alpha_k \in (0,1)\\
%   &0<c_1<c_2<1
% \end{align}

\begin{multicols*}{2}

  \section{Tensor Basics}

  \subsection{Definitions}
  TODO

  \subsection{Tensor Manipulations and Operations}
  TODO

  \vfill\null
  \pagebreak
    
  \section{Tensor Derivatives}

  let $T$ be a contravariant tensor, then:

  $\bar{T}^i = T^r \frac{\partial \bar{x}^i}{\partial x^r}$, where $T=T^i(x(t))$

  Take derivative wrt. $t$:

  $\frac{\partial \bar{T}^i}{\partial t} = \frac{\partial T^r}{\partial t} \frac{\partial \bar{x}^i}{\partial x^r} + T^r \frac{\partial^2 \bar{x}^i}{\partial x^s \partial x^r} \frac{\partial x^s}{\partial t}$ (not a general tensor)

  $\bar{x}^i$ being a linear functions of $x^r \implies \frac{\partial \bar{T}^i}{\partial t} = \frac{\partial T^r}{\partial t} \frac{\partial \bar{x}^i}{\partial x^r}$ (it's a general tensor)
  
  Goal: Would like the curvature of a curve to be an intrinsic concept, independent of coordinate systems. Use Christoffel symbols to solve it.
  
  \subsection{Christoffel Symbol of the 1st Kind}

  $\Gamma_{ijk} = \frac{1}{2}\left( - \frac{\partial (g_{ij})}{\partial x^k} + \frac{\partial (g_{jk})}{\partial x^i} + \frac{\partial (g_{ki})}{\partial x^j} \right)$

  Let $g_{ijk}=\frac{\partial(g_{ij})}{\partial(x^k)}$

  $\Gamma_{ijk} = \frac{1}{2}( -g_{ijk} + g_{jki} + g_{kij})$

  Cyclic permute and add:

  $\Gamma_{ijk} + \Gamma_{jki} = g_{kij}$

  Driving transformation law for $\Gamma_{ijk}$:\\
  By using $g_{ij}$ derivative and its cyclic permutatations,\\
  $\bar{\Gamma}_{ijk} = \frac{1}{2}( -\bar{g}_{ijk} + \bar{g}_{jki} + \bar{g}_{kij})$

  where:
  
  $\bar{g}_{ijk} = \frac{\partial}{\partial \bar{x}^k} \left(g_{rs} \frac{\partial x^r}{\partial \bar{x}^i} \frac{\partial x^s}{\partial \bar{x}^j}\right)$ (expand this)

  Rearrange and simplify to:

  $\bar{\Gamma}_{ijk} = \Gamma_{rst} \frac{\partial x^r}{\partial \bar{x}^i} \frac{\partial x^s}{\partial \bar{x}^j} \frac{\partial x^t}{\partial \bar{x}^k} + g_{rs} \frac{\partial^2 x^r}{\partial \bar{x}^i \partial \bar{x}^j} \frac{\partial x^s}{\partial \bar{x}^k}$ (not a general tensor)
  
  \subsection{Christoffel Symbol of the 2nd Kind}

  $\Gamma_{ij}^k g^{kr} \Gamma_{ijk}$ (raising an index of Christoffel Symbol of the 1st kind.

  Transformation rule for $\Gamma_{ij}^k$; use $\Gamma_{ijk}$ in derivation:
  
  $\bar{\Gamma}_{ij}^k = \bar{g}^{kr} \bar{\Gamma}_{ijr}$

  $\bar{\Gamma}_{ij}^k = \left(g^{sr} \frac{\partial \bar{x}^k}{\partial x^s} \frac{\partial \bar{x}^r}{\partial x^t}\right) \bar{\Gamma}_{ijr}$

  $\bar{\Gamma}_{ij}^k = \left(g^{sr} \frac{\partial \bar{x}^k}{\partial x^s} \frac{\partial \bar{x}^r}{\partial x^t}\right) \left(\Gamma_{uvw} \frac{\partial x^u}{\parital \bar{x}^i} \frac{\partial x^v}{\parital \bar{x}^j} \frac{\partial x^w}{\partial \bar{x}^r} + g_{uv} \frac{\partial^2 x^u}{\partial \bar{x}^i \partial \bar{x}^j} \frac{\partial x^v}{\partial \bar{x}^r} \right)$

  $\bar{\Gamma}_{ij}^k = \Gamma_{uv}^w \frac{\partial \bar{x}^k}{\partial x^w} \frac{\partial x^u}{\partial \bar{x}^i} \frac{\partial x^v}{\partial \bar{x}^j} + \frac{\partial \bar{x}^k}{\partial x^w} \frac{\partial^2 x^w}{\partial \bar{x}^i \partial \bar{x}^j}$ (not a general tensor)

  linear coordinate transform $\implies \Gamma_{ij}^k$ becomes a tensor

  $\frac{\partial^2 x^u}{\partial \bar{x}^i \partial \bar{x}^j} = \bar{\Gamma}_{ij}^k \frac{\partial x^w}{\partial \bar{x}^k} - \Gamma_{uv}^w \frac{\partial x^u}{\partial \bar{x}^i} \frac{\partial x^v}{\partial \bar{x}^j}$
  
  \subsection{Covariant Derivative of Covariant Vector}

  let $T$ be a covariant tensor, then:

  $\bar{T}_i = T_r \frac{x^r}{\partial \bar{x}^i}$

  $\frac{\bar{T}_i}{\partial \bar{x}^k} = \frac{\partial T_r}{\partial \bar{x}^k} \frac{\partial x^r}{\partial \bar{x}^i} + T_r \frac{\partial^2 x^r}{\partial \bar{x}^k \partial \bar{x}^i}$

  use $\frac{\partial^2 x^r}{\partial \bar{x}^i \partial \bar{x}^j}$ from earlier:

  $\frac{\bar{T}_i}{\partial \bar{x}^k} = \frac{\partial T_r}{\partial x^s} \frac{\partial x^s}{\partial \bar{x}^k} \frac{\partial x^r}{\partial \bar{x}^i} + T_r \left( \bar{\Gamma}_{ik}^s \frac{\partial x^r}{\partial \bar{x}^s} - \Gamma_{st}^r \frac{\partial x^s}{\partial \bar{x}^i} \frac{\partial x^t}{\partial \bar{x}^k} \right)$

  $\frac{\bar{T}_i}{\partial \bar{x}^k} = \frac{\partial T_r}{\partial x^s} \frac{\partial x^s}{\partial \bar{x}^k} \frac{x^r}{\partial \bar{x}^i} + \bar{\Gamma}_{ik}^s \bar{T}_s - \Gamma_{st}^r T_r \frac{\partial x^s}{\partial \bar{x}^i} \frac{\partial x^t}{\partial \bar{x}^k}$

  rename indices:

  $\frac{\bar{T}_i}{\partial \bar{x}^k} = \frac{\partial T_r}{\partial x^s} \frac{\partial x^s}{\partial \bar{x}^k} \frac{x^r}{\partial \bar{x}^i} + \bar{\Gamma}_{ik}^t \bar{T}_t - \Gamma_{rs}^t T_t \frac{\partial x^r}{\partial \bar{x}^i} \frac{\partial x^s}{\partial \bar{x}^k}$

  $\frac{\bar{T}_i}{\partial \bar{x}^k} = \bar{\Gamma}_{ik}^t \bar{T}_t + \frac{\partial x^s}{\partial \bar{x}^k} \frac{\partial x^r}{\partial \bar{x}^i} \left( \frac{\partial T_r}{x^s} - \Gamma_{rs}^t T_t \right)$

  $\frac{\bar{T}_i}{\partial \bar{x}^k} - \bar{\Gamma}_{ik}^t \bar{T}_t = \frac{\partial x^s}{\partial \bar{x}^k} \frac{\partial x^r}{\partial \bar{x}^i} \left( \frac{\partial T_r}{x^s} - \Gamma_{rs}^t T_t \right)$

  $\frac{\bar{T}_i}{\partial \bar{x}^k} - \bar{\Gamma}_{ik}^t \bar{T}_t$ is a (0,2)-tensor

  components of covariant derivative wrt. $x^k$ of a covariant vector $T=(T_i)$:

  $T_{,k} = (T_{i,k}) = \frac{\partial T_i}{\partial x^k} - \Gamma_{ik}^t T_t \right)$

  $g_{ij}$ are constants $\implies$ covariant derivatives and partial derivatives coincide

  \subsection{Covariant Derivative of Contravariant Vector}
  
  Similar to previous section, covariant derivative wrt. $x^k$ of a contravariant vector $T=(T^i)$ is:

  $T_{,k} = (T_{,k}^i) = \frac{\partial T^i}{\partial x^k} + \Gamma_{tk}^i T^t$

  Covariant derivative of any rensor is a tensor that has 1 additional covariant order more than the original tensor.

  \subsection{Absolute Differentiation Along Curve}
  
  TODO
  
\end{multicols*}
\end {document}


